\documentclass[12pt,a4paper]{article}

\usepackage{../commons/course}

\hidesolutions


\شروع{نوشتار}

\سربرگ{میان‌ترم اول}{}{زمان آزمون: ۱۶ دی}

\مسئله{شمارنده دودویی}

همان‌طور که قبلاً دیده بودیم هزینه‌ی سرشکن افزایش در یک شمارنده‌ی دودویی از مرتبه‌‌ی $\mathcal{O}(1)$ بود. حالا یک شمارنده دودویی‌ در نظر بگیرید که در آن هزینه تغییر  $i$امین بیت برابر $i$ باشد. ثابت کنید در این حالت نیز بازهم هزینه سرشکن عمل افزایش $\mathcal{O}(1)$ است.

\مسئله{دو پشته}

نشان دهید می‌توان با استفاده از یک آرایه و حافظه‌ی اضافی $O(1)$ دو پشته را پیاده‌سازی کرد. (توجه کنید که زمانی یک پشته نمی‌تواند عمل Push را انجام دهد که کل آرایه پر شده باشد)

\مسئله{بازیابی}

در این مسئله قصد داریم بازسازی درخت‌های دودویی کامل را (درختی که هر گره دقیقا دو فرزند دارد) از روی پیمایش‌های پیش‌وندی و پس‌وندی بررسی کنیم. فرض کنید برچسب گره‌های درخت متمایز هستند.

(آ) پیمایش پیش‌وندی و پس‌وندی یک درخت دودویی کامل در زیر داده شده است. با ذکر دلیل مشخص کنید که آیا می‌توان درخت فوق را ساخت؟ (۶ نمره)

\lr{Preorder: abehdfilcjmnk}

\lr{Postorder: edfhblmnjkcia}

آیا در حالت کلی می‌توان با پیمایش‌های پیش‌وندی و پس‌وندی یک درخت دودویی کامل، درخت را ساخت؟ در صورتی که پاسخ‌تان مثبت است، یک الگوریتم با زمان اجرای چند جمله‌ای ارائه دهید و در غیر این صورت، یک مثال نقض بزنید. (۱۴ نمره)

\مسئله{جدول}

جدولی از اعداد ۰ و ۱ داده شده است که در آن اعداد هر ستون از بالا به پایین به صورت صعودی مرتب شده‌اند. در هر عملیات می‌توانیم مقدار یکی از خانه‌های جدول را بپرسیم.

(آ) با استفاده از
\lr{O(n)}
 عملیات، مکان بالاترین خانه با مقدار ۱ را در جدول
$n \times n$
پیدا کنید.

(ب) همان بحش الف را به ازای جدول شامل
\lr{n}
ستون و
$n^2$
سطر با متوسط
\lr{O(n)}
پرسش حل کنید.

راهنمایی:
در ابتدا یک جایگشت رندوم از ستون‌های جدول ساخته، و به ترتیب آن ستون‌ها را بررسی کنید و مکان بالاترین ۱ را پس از بررسی هر ستون آپدیت کنید.

\مسئله{درخت‌سازی}

تعداد
\lr{n}
جفت عدد داریم، می‌خواهیم این جفت‌ها را تبدیل کنیم به یک درخت باینری، به طوری که اگر به ازای هر جفت عضو اولش را در نظر بگیریم، خروجی یک BST باشد و اگه عضو دومش را در نظر بگیریم، خروجی Heap باشد.

ثابت کنید به ازای هر ورودی، دقیقا یک روش برای این کار وجود دارد. الگوریتمی از اردر
$n^2$
برای ساختن این درخت ارائه دهید و تحلیل پیچیدگی زمان آن را انجام دهید.

\پایان{نوشتار}