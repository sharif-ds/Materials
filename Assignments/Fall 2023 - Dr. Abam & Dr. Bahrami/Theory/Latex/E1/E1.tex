\documentclass[11pt,a4paper]{article}

%\def\hidesols{hide solutions} % uncomment this line to hide solutions
\usepackage{../commons/course}


\شروع{نوشتار}

\سربرگ{تمرین سری اول}{زمان‌ اجرا، تقسیم و حل، سرشکن}{زمان آزمون: ۳ آذر}



 
 \مسئله{رشد توابع}
توابع زیر را بر حسب درجه رشدشان مرتب کنید. 

\begin{center}
\begin{tabular}{ c c c c c }
   $ (\frac{3}{2})^n$ & $ (lgn)! $  &  $ n! $ & $ n^2 $ & $ (\sqrt{2})^{lgn}$ \\
  $ n^{\frac{1}{lgn}} $  & $ 2^{2^n} $  & $ lg(n!) $  & $ n^3  $ & $ lg^2n $ \\
   $ 2^{2^{n+1}} $ & $ lgn$ & $ n^{lg(lg(n))} $ & $ n2^n$ & $ lg(lg(n))$ \\
   $ \sqrt{lgn}$ & $ 2^{\sqrt{2lgn}}$ & $ nlgn $ & $ (lgn)^{lgn}$ & $ 2^{lgn}$ \\
\end{tabular}
 
\end{center}
 \مسئله{مردافکن} 
 دو تابع      $ g : \mathbb{N} \rightarrow \mathbb{N}  $   $ f ,$  بیابید، که اکیدا صعودی باشند و داشته باشیم:   
     $$ g(n) \notin  O(f(n)),  f(n) \notin  O(g(n)) $$

 \مسئله{} 
آرایه‌ی n تایی A  داده شده است. می‌خواهیم از آن ماتریس B  را بسازیم که در آن    $ B[i,j] = \sum_{k=i}^j{A[k]} $  باشد (برای $i \le j $ ).   اگر $ i > j $  مقدار $ B[i,j] $ مهم نیست. 
\شروع{ابجد}


\فقره
الگوریتم زیر را برای محاسبه‌ی B  پیشنهاد می‌کنیم: 
\latin
%
\begin{algorithmic}[0]

\For{$i \leftarrow 1 \hspace{4pt} to \hspace{4pt} n$} 

\For{$j \leftarrow i \hspace{4pt} to \hspace{4pt} n$} 

\State $B[i,j] = \sum_{k=i}^j{A[k]}$

\end{algorithmic}


\persian

دقیقا چه تعداد عمل جمع در این الگوریتم انجام می‌شود؟ 
\فقره
الگوریتمی با تعداد بهینه جمع ارائه دهید. این تعداد دقیقا چقدر است؟ 



\پایان{ابجد}


\مسئله{رشد عجیب} 
فرض کنید توابع $ f $  و  $ g $ به گونه ای داده شده اند که $ f(n) \in O(g(n)) $. برای هر یک از گزاره‌های  زیر درستی و نادرستی آن‌ها را با دلیل ثابت کنید. (برای اثبات نادرستی، مثال نقض کافی است)

\شروع{ابجد}

\فقره
$ log(f(n)) \in O(log(g(n))) $
\فقره
$ 2^{f(n)} \in O(2^{g(n)}) $ 
\فقره
$ f(n)^2 \in O(g(n)^2) $ 

\پایان{ابجد}

\مسئله{}
ثابت کنید: $ \sum_{i=1}^{n}\sqrt{i} \in \Theta(n\sqrt{n}) $ 

\مسئله{بازگشتی}


روابط بازگشتی زیر را حل کنید.
\شروع{ابجد}

\فقره
 
 $T(n) = T(\frac{n}{2})+\frac{n}{\log{n}}$
\فقره 
 $T(n) =2T(\frac{n}{2})+\frac{n}{\log{n}}$

 
 \پایان{ابجد}

\مسئله{دنباله‌ی طلایی}


 $a_1$ تا $a_k$، $k$ عدد حقیقی بزرگتر از یک می‌باشند. برای $a_i$ ها یک شرط لازم و کافی پیدا کنید به طوری که $T(n) = \sum_{i=1}^{k} T(\frac{n}{a_i})+\Theta(\frac{n}{\log{n}})$ از مرتبه‌ی $\Theta(\frac{n}{\log{n}})$ باشد.

\مسئله{بازگشت عجیب}

تابع 
$T:\mathbb{N}\rightarrow\mathbb{R}^+$
توسط رابطه‌ی بازگشتی زیر داده شده است:
$$T(n) = 
\left\{\begin{matrix}
a &  n = 1\;\mbox{اگر}\\ 
bn^2 + nT(n-1) & \mbox{\rl{در غیر این‌صورت}}
\end{matrix}\right.
$$

که $a,b$ اعداد حقیقی و مثبت‌اند.
\شروع{ابجد}
\فقره
ثابت کنید $T(n) \in \Theta(n!)$.
\فقره
رابطه‌ی دقیق و صریح $T(n)$  را بیابید. این رابطه را برحسب $a$ و $b$ و $c$ بنویسید که 
$$c = \lim_{n\rightarrow\infty}{\frac{T(n)}{n!}}$$
همچنین بررسی کنید 
$a\leq c\leq a+5b$.
\فقره
فرض کنید تابع $g:\mathbb{N}\rightarrow\mathbb{R}^+$ با این رابطه داده شده باشد:
$$g(n) = 
\left\{\begin{matrix}
a &  n = 1\;\mbox{اگر}\\ 
bn^k + ng(n-1) & \mbox{\rl{در غیر این‌صورت}}
\end{matrix}\right.
$$
ثابت کنید $g(n) \in \Theta(n!)$.
\پایان{ابجد}


\مسئله{حدس پیچیده}


در هر قسمت، برای $T(n)$ بهترین مرتبه‌ی ممکن را بیابید.
\شروع{ابجد}
\فقره
 $T(n) =2T(\frac{n}{2})+1$

\فقره
 $T(n) =2T(\frac{n}{2}+\sqrt{n})+T(\frac{n}{2})+1$

\پایان{ابجد}

\مسئله{شمارنده دودویی}


همان‌طور که قبلاً دیده بودیم هزینه‌ی سرشکن افزایش در یک شمارنده‌ی دودویی از مرتبه‌‌ی $\mathcal{O}(1)$ بود. حالا یک شمارنده دودویی‌ در نظر بگیرید که در آن هزینه تغییر  $i$امین بیت برابر $i$ باشد. ثابت کنید در این حالت نیز بازهم هزینه سرشکن عمل افزایش $\mathcal{O}(1)$ است.

\مسئله{حذف پر هزینه}



فرض کنید $n$ عدد دودویی دارید که در ابتدا همه‌ی آن‌ها برابر یک هستند. در هر مرحله دو عدد دلخواه را انتخاب کرده و از مجموعه حذف می‌کنیم و به جای آن‌ها حاصل جمعشان را قرار می‌دهیم. اگر دو عددی که حذف کردیم $b_1$ و $b_2$ بیتی باشند، هزینه‌ی این عمل برابر است با:


$min(b_1,b_2)$ به علاوه‌ی تعداد بیت‌های نقلی در جمع که بعد از بیت سمت چپ عدد کوچکتر به وجود می‌آید. مثلا هزینه‌ی جمع دو عدد 1100 و 10110100 برابر است با 7 = 3 + 4 و هزینه‌ی جمع دو عدد 101 و 100001  برابر 3 است. حال ثابت کنید اگر $m$ بار این عمل را انجام دهیم حداکثر از $\mathcal{O}(m)$ هزینه صرف کرده‌ایم.



\مسئله{آرایه‌ی جادار}

می‌خواهیم $n$ عدد را به ترتیب، در انتهای آرایه‌ای اضافه کنیم. طول آرایه در ابتدا ۱ است. در نوبت اضافه‌کردن یک عدد به انتهای آرایه، اگر آرایه فضای خالی داشت، عدد را در انتها اضافه می‌کنیم. در غیر این صورت، آرایه‌ای جدید به طول دوبرابر آرایه فعلی ایجاد می‌کنیم، عناصر را از آرایه قبلی به آرایه‌ی جدید منتقل می‌کنیم و سپس عنصر جدید را در انتهای آرایه اضافه می‌کنیم. پیچیدگی زمانی هر عمل اضافه کردن را محاسبه کنید.

\مسئله{کار و بار}

تعداد نامعلومی کار باید انجام شود. اگر $i$ به صورت توانی از 2 بود، انجام کار $i$ام هزینه‌ای برابر با $i$ خواهد داشت و در غیر این صورت هزینه‌ی آن کار ۱ است. با سه روش الف) انبوهه، ب) حسابداری و ج) تابع پتانسیل
ثابت کنید که هزینه‌ی سرشکن هر کار $O(1)$ می‌باشد.

\پایان{نوشتار}
