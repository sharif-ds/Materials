\documentclass[11pt,a4paper]{article}

%\def\hidesols{hide solutions} % uncomment this line to hide solutions
\usepackage{../commons/course}


\شروع{نوشتار}

\سربرگ{تمرین سری چهارم }{درخت دودویی جستجو و مجموعه‌های مجزا}{زمان آزمون: ۵ دی}


\مسئله{}

ثابت کنید در زمان $o(n\log n)$ نمی‌توان $n$ عدد را بصورت درخت دودویی جستجو درآورد.

\مسئله{}

مسئله درخت دودویی جستجوی (د.د.ج) بهینه با $n$ عدد را در نظر بگیرید.  در مسئله د.د.ج بهینه $n$ عدد  به همراه تعداد  دفعاتی که پرسمان خواهند شد داده می‌شود. هدف ساخت یک د.د.ج است به گونه‌ای که مجموع حاصلضرب پرسمان اعداد در عمق آن‌ها در د.د.ج کمینه شود.
الگوریتم حریصانه زیر را در نظر بگیرید.
عدد با بیشترین پرسمان را در ریشه درخت قرار می‌دهیم.  براساس ریشه مشخص شده  اعداد باقی‌ماده براساس خاصیت د.د.ج در یکی از  زیردرخت‌های  چپ یا راست قرار می‌گیرند. بصورت بازگشتی زیردرخت  چپ و راست را می‌سازیم. کوچکترین $n$ی که این الگوریتم حریصانه درست کار نمی‌کند چند است.


\مسئله{}

چند درخت دودویی جستجوی متفاوت با $n$ گره و برچسب‌های $1$ تا $n$ وجود دارد 
طوری که پیمایش پیش‌ترتیب و میان‌ترتیب آن‌ها یکسان باشد؟ ٔدلیل خود را ذکر کنید.

\مسئله{}

اعداد ۱ تا ۵۰۰ را در یک درخت دودویی جستجو ذخیره کرده‌ایم. می‌خواهیم عدد ۱۹۳ را در این درخت جستجو کنیم. کدام یک از دنباله‌های زیر نمی‌تواند مسیر جستجو برای عدد ۱۹۳ باشد. دلیل خود را ذکر کنید.
\شروع{شمارش}

\فقره  $5, 454, 300, 100, 250, 150, 200, 193$
\فقره $500, 400, 300, 200, 50, 100, 150, 193$
\فقره $437, 157,237, 231, 201, 143 190, 193$
\فقره $4, 20, 30, 55, 101, 102, 105, 177, 193$

\پایان{شمارش}


\مسئله{}

فرض کنید یک درخت دودویی جست‌وجو با $n$ گره داریم. به ازای گره $v$ از این درخت وزن آن را تعداد گره‌ها در زیر درخت به ریشه $v$ (شامل $v$) در نظر بگیرید. می‌دانیم در درخت فوق به ازای هر گره‌ی داخلی $v$ نسبت وزن فرزند چپ و فرزند راست حداقل ۰.۵ و حداکثر ۲ است. بهترین کران بالا  برای زمان جست‌وجو در این درخت در بدترین را محاسبه کنید.

\مسئله{}

یک درخت دودویی جست‌وجو متوازن با $n$ گره داریم که به علت نویز، اعداد ذخیره شده در برخی از گره‌های آن  تغییر کرده است. تنها عملی که می‌توان برای اصلاح این درخت انجام داد جابه‌جا کردن مقادیر ذخیره شده در یک گره  و یکی از فرزندان آن است. در بدترین حالت  با چند عمل فوق می‌توان درخت را به درخت دودویی جست‌وجو معتبر تبدیل کرد.



\مسئله{}

درستی یا نادرستی جملات زیر  در مورد درخت دودویی جست‌وجو  (د.د.ج) را با ذکر دلیل مشخص کنید.

\شروع{شمارش}

\فقره اگر یک عنصر موجود در د.د.ج را حذف و بلافاصله درج کنیم، د.د.ج قبل و بعد از دو عمل فوق یکسان است.
\فقره هر د.د.ج را می‌توان با چند عمل چرخش (rotation) به یک د.د.ج متوازن تبدیل کرد.
\فقره عدد بلافاصله بعد از  $x$  در ترتیب صعودی، لزوما در زیردرخت به ریشه گره‌ای که  $x$ در آن ذخیر شده قرار نمی‌گیرد. 
 
 \پایان{شمارش}

\مسئله{}

فرض کنید یک درخت دودویی با $n$ گره داده شده است. درخت لزومن متوازن نیست. به ازای هر گره $u$ از درخت، اندازه دو زیردرخت سمت چپ و راست آن  را محاسبه کرده و مینیمم این دو را  به عنوان برچسب گره $u$ در نظر می‌گیریم. منظور از اندازه یک زیردرخت تعداد گره‌های آن می‌باشد. اگر زیردرختی تهی باشد اندازه آن را صفر در نظر می‌گیریم.  نشان دهید مجموع برچسب‌ها از مرتبه $O(n\log n)$ است.



\مسئله{}


درستی عبارات زیر را با ذکر دلیل مشخص کنید.
\شروع{شمارش}
 \فقره  گره‌های هر درخت دودویی جست و جو را می‌توان با رنگ‌های قرمز و سیاه رنگ کرد طوری که درخت حاصل قرمز-سیاه شود.
 \فقره گره‌های هر درخت دودویی جست و جو با $n$ عنصر و ارتفاع حداکثر $2\log n$ را می توان با رنگ‌های قرمز و سیاه رنگ کرد طوری که درخت حاصل قرمز-سیاه شود.
 \فقره  یک درخت دودویی جست و جوی کاملا متوازن را می‌توان با رنگ‌های قرمز و سیاه رنگ کرد طوری که درخت حاصل قرمز-سیاه شود.
\پایان{شمارش}
 
 
 \مسئله{}

در درخت بازه توضیح داده شده در کلاس، فرض کنید می‌خواهیم به ازای عدد داده شده $x$ تمام بازه‌هایی که نقطه $x$  را شامل می‌شوند گزارش دهیم. چه تغییرات در روال پاسخ‌دهی به پرسمان باید ایجاد کنیم. زمان پاسخگویی به پرسمان را برحسب $n$ و $k$ مشخص کنید که  $n$ و $k$ به ترتیب تعداد بازه‌های ذخیره شده در درخت بازه و تعداد بازه‌هایی که $x$ را شامل می شوند می‌باشند.

\مسئله{}

فرض کنید $b_n$ برابر تعداد درخت‌های دودویی متفاوت با $n$ راس باشند؛ برای مثال $b_0 = b_1 = 1$ و $b_2 = 2$ است. رابطه‌ای بازگشتی برای $b_n$ ارائه دهید و به کمک آن فرمول صریحی برای آن پیدا کنید.

\مسئله{}

اعداد صحیح $x_1, \cdots, x_n$ را در یک درخت دودویی جستجو با ارتفاع $h$ ذخیره کرده‌ایم. فرض کنید هزینه‌ی جستجوی $x_i$ (تعداد مقایسه‌های لازم در درخت برای پیدا کردن $x_i$) برابر $c_i$ باش‌. می دانیم $\sum_{i=1}^n c_i = O(n\log n)$ است. برای درستی و یا نادرستی عبارات زیر دلیل بیاورید.

\شروع{ابجد}
    \فقره
    $h = O(\log n)$
    
    \فقره
    $h = O(\sqrt{n \log n})$
    
    \فقره
    می‌توان مثالی زد که $h=\Omega(n)$ باشد.
    
    \فقره
    $h = \Omega(\sqrt{n})$
\پایان{ابجد}
  
\مسئله{}

یک درخت دودویی جستجو به ارتفاع $h$ داریم. فرض کنید اعداد $x_1 \leq x_2 \leq \cdots \leq x_n$ در آن ذخیره شده‌اند. می‌خواهیم مقدار عدد $x_k$ را به وسیله‌ی درخت پیدا کنیم. الگوریتمی از مرتبه زمانی $O(h+k)$ ارائه دهید که بتواند مقدار این عدد را به وسیله‌ی درخت پیدا کند. توجه داشته باشید که ما به ازای هر راس فقط به بچه‌ی سمت راست و چپ آن (در صورت وجود) دسترسی داریم و حق نگه‌داری پارامتر دیگری در درخت خود به ازای رئوس مختلف در روند اضافه کردن مقادیر به درخت نداریم.

%\مسئله{}

%یک گراف $n$ راسی $m$ یالی داریم. $q$ مرحله داریم که در هر مرحله یک یال بین دو راس $u_i$ و $v_i$ به گراف اضافه می‌شود. الگوریتمی از مرتبه زمانی $O(n \log(n) + q)$ ارائه دهید که بعد از اضافه کردن یال موردنظر در هر مرحله، دوبخشی بودن یا نبودن گراف جدید را بررسی کند. الگوریتم شما باید به صورت برخط (Online) عمل کند و تا خروجی ندادن جواب مرحله‌ی $i$ ام، اعداد مرحله‌ی $i+1$ ام به شما داده نمی‌شود.

%\مسئله{}

%یک گراف $n$ راسی $m$ یالی داریم که هر یال آن نیز یک وزن مشخصی دارد. الگوریتمی از مرتبه زمانی $O(m \log(m))$ ارائه دهید که به ازای هر  یال در گراف، مشخص کند که این یال:

%\begin{itemize}
%	\item همواره در زیردرخت القایی کمینه حضور دارد.
%	\item ممکن است در زیردرخت القایی کمینه حضور داشته باشد.
%	\item هیچ‌گاه در زیردرخت القایی کمینه وجود نخواهد داشت.
%\end{itemize}

\پایان{نوشتار}