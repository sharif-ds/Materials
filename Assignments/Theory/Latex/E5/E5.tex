\documentclass[11pt,a4paper]{article}

%\def\hidesols{hide solutions} % uncomment this line to hide solutions
\usepackage{../commons/course}


\شروع{نوشتار}

\سربرگ{تمرین سری پنجم }{درهم‌سازی}{زمان آزمون: ۲۴ دی}

\مسئله{}

فرض کنید برای درهم‌سازی $n$ کلید متمایز از روش درهم‌سازی زنجیره‌ای استفاده شده است. تابع درهم‌سازی ساده و یکنوا و اندازه آرایه $m$  می‌باشد. امید ریاضی تعداد برخوردها (تعداد جفت کلیدهایی که به یک خانه نگاشت می‌شوند) برحسب $m$ و $n$ چه پیچدگی محاسباتی دارد؟

\مسئله{}

فرض کنید از روش آدرس‌دهی باز با استفاده از وارسی خطی برای درهم‌سازی استفاده شده است. اندازه جدول درهم‌سازی 10 و تابع درهم ساز $h(x) = x^2 \bmod 10$ است. با فرض آنکه ورودی‌ها به ترتیب $5, 17, 53, 123, 37 , 52 , 49 , 30 , 1 , 19$ (از چپ به راست) باشد، اعداد به چه ترتیب در جدول ذخیره می‌شوند.

\مسئله{}

وضعیت فعلی یک جدول درهم‌ساز در زیر آمده است. فرض کنید برای رفع مشکل تصادم از روش وارسی خطی استفاده شده است. با در نظر گرفتن فرض یکنواختی تابع درهم‌ساز، کلید بعدی با چه  احتمالی در خانه‌ی دوم قرار می‌گیرد؟ (خانه‌های جدول از چپ به راست از ۱ تا ۱۸ شماره‌گذاری شده‌اند.)

$$H[1..18]=\{6,-, 1,- ,3 ,- , 14, -, 9, 2, -, 11, -, -, -, 0, 4, 5\}$$


\مسئله{}

فرض کنید از آدرس‌دهی باز و وارسی خطی برای درهم‌سازی استفاده شده  و تابع درهم سازی $i^2$ به پیمانه ۷ است. بعد از دریافت همه اعداد $0,\dots, 6$ می‌دانیم نحوه‌ی قرار گیری اعداد در جدول درهم‌ساز به صورت زیر است:

$$A[0,..,6] = 0,6, 4, 3, 1,5,2$$

به ازای چند جایگشت ورودی وضعیت جدول درهم‌ساز به شکل بالا خواهد بود.


\مسئله{}

جدول ‌درهم‌سازی ۱۰ خانه‌ای و  تابع درهم‌ساز $h(x)= 3x+5 \mod 10$ را در نظر بگیرید. کدام گزینه درست است. توضیح دهید.

\شروع{شمارش}
\فقره احتمال آن که ورودی $x=4$  به خانه 7 نگاشت شود برابر $1/10$ است.
\فقره احتمال آن که ورودی $x=4$  به خانه 7 نگاشت شود برابر $0$ است.
\فقره احتمال آن که ورودی $x=4$  به خانه 7 نگاشت شود برابر $1$ است.
\فقره احتمال آن که دو ورودی مختلف به یک خانه نگاشت شوند برابر $1/10$ است.
\پایان{شمارش}
  
  
\مسئله{}

اگر اعداد ۱ تا $n$ را به ترتیب تصادفی در یک درخت جست‌وجوی دودویی درج کنیم،  رابطه‌ی بازگشتی امید ریاضی  ارتفاع این درخت را بدست آورید.


\مسئله{}

فرض کنید $H: \{1, \cdots, n\} \rightarrow \{1,\cdots, n\}$ یک تابع درهم‌ساز یکنواخت باشد. برای ورودی $x$، عدد $z$ را برابر تعداد صفرهای سمت راست $H(x)$ قرار می‌دهیم. برای عدد $0 \leq c \leq 1$، احتمال $z\geq c\log n$ از چه مرتبه‌ای‌ است؟ فرض کنید $c$ ثابت است.



.

\مسئله{}

الگوریتمی را در نظر بگیرد که ورودی $a_1, \dots, a_n$ شامل $n$ عدد مجزا را به ترتیب داده‌شده می‌خواند و هنگام خواندن $a_i$ مقدار متغیر $x$ را به احتمال $1/i$ برابر $a_i$  قرار می‌دهد. الگوریتم در پایان مقدار $x$ را به عنوان خروجی گزارش می‌کند. با چه احتمالی خروجی الگوریتم برابر $a_i$ است؟

 
 \مسئله{}

ثابت کنید اگر
H
یک خانواده 
۲-سراسری از توابع درهم‌ساز باشد،
یک خانواده سراسری نیز خواهد بود.

 \مسئله{}


فرض کنید می‌خواهیم درون یک جدول درهم‌سازی با خانه‌های 
$\{0,1,...,m-1\}$
دنبال عنصر داده‌شده‌ی $k$ بگردیم. همچنین تابع درهم‌سازی 
$h: \mathcal{I} \rightarrow \{0,1,...,m-1\}$
که $\mathcal{I}$ نمایش‌دهنده‌ی فضای عناصر می‌باشد را در اختیار داریم. روش جست‌و‌جوی ما در زیر آمده است:\\
۱) مقدار 
$i\leftarrow h(k)$
را محاسبه کن و 
$j\leftarrow 0$
قرار بده.\\
۲) در درایه‌ی $i$ به دنبال $k$ بگرد. اگر آن‌ را یافتی یا اگر درایه خالی بود جست‌وجو را متوقف کن.\\
۳) $j\leftarrow (j+1)\ mod\ m$
و 
$i\leftarrow (i+j)\ mod\ m$
قرار بده و به مرحله‌ی ۲ برگرد.\\
فرض کنید $m$ توانی از ۲ است.\\
\شروع{شمارش}
\item
نشان دهید که این روش یک روش از روش کلی وارسی درجه ۲ است. این کار را با تعیین مقدار‌های مناسب برای ثابت‌های $c_1$ و $c_2$ انجام دهید.

\item
ثابت کنید این الگوریتم در بدترین حالت هر درایه از جدول را وارسی می‌کند.

\پایان{شمارش}

 \مسئله{}
فرض کنید $\sigma$ ‌یک چایگشت تصادفی از اعداد ۱ تا $n$ باشد. امیدریاضی تعداد $i$ هایی را بدست آورید که ${\sigma}_i=i$.

\newpage

\مسئله{}

فرض کنید $\sigma$ یک چایگشت تصادفی از اعداد ۱ تا $n$ باشد. امیدریاضی تعداد وارونه‌ها را بدست آورید. یک وارونه، یک زوج $(i,j)$ است بطوری که $i< j$ اما $\sigma(i)> \sigma(j)$.

\مسئله{}

یک رشته از حروف انگلیسی به طول $n$ داریم. به یک زیررشته از آن «پالیندروم» گفته می‌شود، اگر و تنها اگر آن رشته از دو طرف (چپ به راست و راست به چپ) به صورت یکسان خوانده شود. برای مثال رشته‌ی abba پالیندروم است، ولی رشته‌ی abaa پالیندروم نیست. الگوریتمی از مرتبه زمانی $O(n \times \log(n))$ ارائه دهید که تعداد تمام زیررشته‌های پالیندروم رشته‌ی اصلی را پیدا کند (اگر زیررشته‌ای چندین بار در رشته‌ی اصلی تکرار شده بود، آن را به تعداد تکرارهایش می‌شماریم).

\مسئله{}

به دو درخت $T_1$ و $T_2$ یک‌ریخت می‌گوییم، اگر و تنها اگر بتوان این دو درخت را از راسی ریشه‌دار کرد و سپس به بچه‌های هر راس ترتیبی داد که دو درخت دقیقاً یکسان شوند. الگوریتمی از مرتبه زمانی $O(n \times \log(n))$ ارائه دهید که برای دو درخت داده‌شده‌ی $T_1$ و $T_2$ با سایز برابر $n$ مشخص کند که آیا این دو درخت یک‌ریخت هستند یا خیر.

\مسئله{}

دنباله‌ی $a_1, \dots, a_n$ داده شده‌است که $1 \leq a_i \leq k$ است. به یک دنباله از اعداد که تعداد تکرار تمام اعداد $1$ تا $k$ در آن برابر باشد (یعنی تعداد $1$ های دنباله برابر تعداد $2$ های دنباله، برابر تعداد $3$ های دنباله، ... و برابر تعداد $k$ های دنباله باشد) طلایی می‌گوییم. الگوریتمی از $O(n)$ ارائه دهید که تعداد زیردنباله‌های متوالی طلایی دنباله‌ی $a$ را محاسبه کند.

\مسئله{}

رشته $A$ را \textbf{ریشه} یک رشته دیگر مانند $B$ می‌گوییم اگر $A$ یک جایگشت از حروف رشته $B$ باشد. به طور مثال رشته‌های $(\text{indicatory}, \text{dictionary})$ و $(\text{brush}, \text{shrub})$ زوج مرتبی از کلمات هستند که ریشه یک‌دیگر اند. در این مسئله تمام رشته‌ها به صورت $ASCII$ می‌باشند و فقط شامل حروف انگلیسی کوچک از $a$ تا $z$ هستند. دو رشته $A$ و $B$ داده شده است، \textbf{تعداد زیررشته ریشه} برای رشته $B$ در $A$  برابر با است با تعداد زیررشته‌های مجاور از $A$ به طوری که ریشه $B$ هستند. به طور مثال، اگر $A = \text{'esleastealaslatet'}$ و $B = \text{'tesla'}$ باشد، در این صورت سه‌تایی مرتب $(\text{'least'}, \text{'steal'}, \text{'slate'})$ را در نظر بگیرید و خواهیم داشت تعداد زیررشته‌های ریشه از $B$ در  $A$ برابر با 3 خواهد بود.

\شروع{ابجد}
\item
رشته $A$ و عدد طبیعی k داده شده است. یک داده ساختار ارائه دهید که می‌تواند در زمان $O(|A|)$ ساخته شود و اگر یک رشته دیگر مانند $B$ به طوری که $|B| = k$ باشد داده شود، بتواند تعداد زیررشته‌های ریشه $B$ را در $A$ با زمان $O(k)$ بیابد.

\item
رشته $T$ و آرایه‌ای متشکل از $n$ رشته که هر یک به طول $k$ هستند به صورت $S = (S_0, ..., S_{n-1})$ را در نظر بگیرید. اگر $0 < k < |T|$ برقرار باشد، یک الگوریتم با زمان $O(|T| + nk)$ ارائه دهید که آرایه‌ای به صورت $A = (a_0, ..., a_{n-1})$ که هر $a_i$ تعداد زیررشته‌های ریشه $S_i$ در $T$ است را به ازای تمام $i \in \{0, \ldots, n-1\}$ برگرداند.
\پایان{ابجد}

\پایان{نوشتار}

\مسئله{}

دنباله‌ی $a_1, \dots, a_n$ و عدد $k$ که $1 \leq k \leq n$ داده شده است. به زوج مرتب $(i, j)$ که $1 \leq i \leq j \leq n-k+1$ است \textbf{برنده} می‌گوییم، اگر رابطه‌ی زیر برای آن‌ها برقرار باشد:
$$
a_i + a_j = a_{i+1} + a_{j+1} = \dots = a_{i+k-1} + a_{j+k-1}
$$
الگوریتمی از $O(n)$ ارائه دهید که تعداد زوج‌های مرتب برنده را بشمارد.

\مسئله{}

به یک چهارتایی $(a,b,c,d)$ از اعداد حقیقی که $\sqrt{a^2+b^2+c^2}=d$ است، فیثاغورثی می‌گوییم. یک آرایه‌ی $n$ عضوی از اعداد حقیقی مثبت به عنوان ورودی داده می‌شود. الگوریتمی از مرتبه‌ی زمانی $O(n^2)$ ارائه دهید که تعداد چهارتایی‌های فیثاغورثی را در دنباله‌ی ورودی حساب کند.

\مسئله{}

فرض کنید $A$ آرایه‌ای از اعداد طبیعی به طول $n$ باشد. به شما $q$ پرسش به صورت $(l,r)$ داده می‌شود که از شما می‌پرسد «آیا تمامی اعداد در این بازه، زوج بار تکرار شده اند یا نه؟». به عبارتی دیگر، شرط زیر برای این بازه‌ی $(l,r)$ از دنباله برقرار باشد:
$$\forall x \in \mathbb{N}: cnt_{l,r}(x) \mod 2 = 0$$
در عبارت بالا منظور از $cnt_{l,r}(x)$ تعداد اعداد $x$ در بازه‌ی $(l,r)$ است. الگوریتمی از مرتبه زمانی $O(n + q)$ ارائه دهید که پاسخ هر پرسش را مشخص کند.