\documentclass[12pt,a4paper]{article}

\usepackage{../commons/course}

\hidesolutions

\شروع{نوشتار}

\سربرگ{مسائل نظری سری ششم}{گراف}{زمان آزمون: ۲۴ دی}

\مسئله{کوتاه کردن تدریجی}

فرض کنید در یک گراف وزن‌دار (منفی یا مثبت) که وزن همه‌ی دورها در آن مثبت است می‌خواهیم کوتاه‌ترین فاصله از راس $s$ به بقیه راس‌ها را محاسبه کنیم. برای این کار در ابتدا $d(s)=0$  و $\forall u \neq s: d(u)=+\infty$ می‌گذاریم. هر بار به دلخواه یک یال $(u,v)$ را که $d(u)+ w(u,v) < d(v)$ انتخاب کرده و مقدار $d(v)$ را با مقدار $d(u)+w(u,v)$ بروزرسانی می‌کنیم که $w(u,v)$ وزن یال $(u,v)$ می‌باشد. درستی الگوریتم فوق را اثبات و چندجمله‌ای بودن زمان اجرای آن در بدترین حالت را بررسی کنید.

\مسئله{جست‌وجو در ژرف}

فرض کنید یک گراف ۵ راسی همبند داریم که راس‌های آن با شماره‌های ۱ تا ۵ شماره‌گذاری شده‌اند. فرض کنید از راس ۱  DFS را اجرا می‌کنیم. فرض کنید تمام حالت‌هایی که DFS‌ می‌تواند رئوس را ملاقات کند عبارتند از $<1, 2, 4 , 3 , 5>$، $<1, 3, 4, 2 , 5>$ و $<1 , 3, 5, 4, 2>$. حال اگر از راس $5$  DFS را اجرا کنیم ترتیب ملاقات‌ها به چه شکل می‌تواند باشد. دلیل خود را بیان کنید.

\مسئله{دوبینی}

فرض کنید یک گراف بدون‌جهت داریم که هر یال آن دارای دو وزن مثبت است. بار اولی که از یک یال عبور می‌کنیم باید به اندازه وزن بیش‌تر آن یال هزینه پرداخت کنیم و بارهای بعدی به اندازه وزن سبک‌تر هزینه پرداخت می‌کنیم.  می‌خواهیم از  راس $u$ به راس $v$ برویم و در مسیر از راس $w$ عبور کنیم. الگوریتمی از مرتبه $O(n\log n+m)$ ارائه دهید که مسیر با کم‌ترین وزن را پیدا کند که $n$ و $m$ به ترتیب تعداد رئوس و تعداد یال‌های گراف می‌باشند.

\مسئله{کوتاه‌ترین مسیر}

فرض کنید گراف $G$ یک گراف وزن‌دار همبند با $n$ راس باشد که دارای دور منفی نیست. رئوس گراف را به ترتیب دلخواه از $1$ تا $n$ شماره‌گذاری می‌کنیم. فرض کنید $d(u,v)$ برابر طول کوتاهترین مسیر از $u$ به $v$ باشد. به ازای عدد دلخواه $k$ ($0\leq k \leq n$) دو تابع فاصله زیر را تعریف می‌کنیم.

\شروع{شمارش}
\فقره $g^k(u,v)$ برابر طول کوتاه‌ترین مسیر از $u$ به $v$ که تعداد یال‌های مسیر حداکثر $k$ باشد.
\فقره $h^k(u,v)$ برابر طور کوتاه‌ترین مسیر از $u$ به $v$ که شماره‌ی راس‌های میانی (به غیر از $u$ و $v$) حداکثر $k$ باشد.
\پایان{شمارش}

ابتدا نشان دهید $g^n(u,v)=h^n(u,v)=d(u,v)$ است. سپس مقدار $g^1(u,v)$ و $h^{0}(u,v)$ را محاسبه کنید. نهایتا برای هر دو تابع $h$ و $g$ یک رابطه بازگشتی بنویسید.

\newpage

\مسئله{رنگ‌آمیزی}
دور یک دایره اعداد یک تا $n$ را طوری نوشته‌ایم که هر عدد دقیقاً دوبار آمده است. می‌خواهیم با دو رنگ تمام اعداد را رنگ‌آمیزی کنیم، به طوری که هیچ سه عدد متوالی‌ای همرنگ نشوند و هر دو عددی که مقدارشان یکسان است نیز ناهمرنگ شوند. الگوریتمی ارائه دهید که در مرتبه زمانی $O(n)$ این کار را انجام دهد.
	
\مسئله {بی‌اف‌اس کم‌عمق}

در گراف همبند $G=(V,E)$ شامل $n$ راس، اگر از هر راس BFS را اجرا کنیم ارتفاع درخت BFS حداکثر ۲ می‌شود. نشان دهید برای هر $n-1 \leq i \leq n(n-1)/2$ می‌توان گرافی با $i$ یال مثال زد که این ویژگی را داشته باشد. 

\مسئله{ترتیب ملاقات}
 فرض کنید رئوس گراف همبند و  بدون جهت $G$ با اعداد $1, 2, \dots, n$ شماره‌گذاری شده‌اند. ار راس شماره $1$ الگوریتم BFS را اجرا کرده‌ایم و ترییب ملاقات رئوس از چب به راست به ترتیب $1, 2, \dots, n$ شده است. درستی گزاره‌های زیر را با ذکر دلیل مشخص کنید.
\شروع{شمارش}
 \فقره بین راس $i$ و $i+1$ به ازای هر $i$ یال وجود دارد.
  \فقره از راس $n$ می‌توان به گونه‌ای BFS را اجرا کرد که ترتیب ملاقات رئوس $n,\dots, 2, 1$ شود.
  \فقره به ازای هر $i>1$ حتما یک $j<i$ وجود دارد که بین $i$ و $j$ یک یال وجود دارد.
\پایان{شمارش}

\مسئله{تبدیل ارز}

فرض کنید نرخ تبدیل $n$‌ ارز موجود به یکدیگر را می‌دانیم. $m$  ریال پول در اختیار داریم. می‌خواهیم بدانیم با چندین بار تبدیل پول و نهایتا تبدیل آن به ریال می‌توانیم مقدار $m$ را افزایش دهیم. الگوریتمی با زمان اجرای چندجمله‌ای برای تشخیص چنین کاری ارائه دهید.


\مسئله{میزبانی جام‌ ملت‌های آسیا}

قرار است میزبان جام ملت‌های آسیا دوره‌ی بعد بزودی مشخص شود.  لیست نامزدها مشخص است و کنفدراسیون فوتبال آسیا (ای‌اف‌سی) بررسی‌های  لازم خود را از کشورهای نامزد  انجام داده است. باتوجه به بررسی‌های انجام شده، درحال حاضر مشخص است اگر ای‌اف‌سی بخواهد بین دو   کشور نامزد $A$ و $B$ یکی را انتخاب کند کدام کشور را انتخاب خواهد کرد. سیستم انتخاب میزبان توسط ای‌اف‌سی بدین‌شکل است. در هر مرحله از بین نامزدهای باقی‌مانده، دو  نامزد را بطور کاملا تصادفی انتخاب می‌کند و نامزدی که رای ای‌اف‌سی با او نیست را حذف می‌کند. با فرض آنکه نظر ای‌اف‌سی را در مورد هر دو کشور نامزد می‌دانیم، می‌خواهیم کشورهایی که شانس کسب میزبانی را دارند را پیدا کنیم.  برای این کار یک گراف جهت‌دار $n$ راسی  می‌سازیم که $n$ تعداد کشورهای نامزد است و هر راس متناظر با یک کشور نامزد است. برای هر دو راس $A$ و $B$ یک یال بین آن‌ها می‌گذاریم و جهت یال را به سمت کشوری می‌گذاریم که نظر ای‌اف‌سی با آن کشور است. 
\شروع{شمارش}
\فقره نشان دهید کشور $A$ شانس میزبانی دارد اگر و فقط اگر از همه‌ی رئوس به راس $A$ مسیر وجود داشته باشد.
\فقره الگوریتمی با زمان اجرای $\Theta(n^2)$ ارائه دهید که تمام کشورهایی که شانس میزبانی را دارند را پیدا کند. دقت کنید که تعداد یال‌های گراف از  $\Theta(n^2)$ است.
\پایان{شمارش}

\مسئله{بلندترین مسیر کوتاه}
گراف جهت‌دار $G = (V, E)$ با وزن‌های دلخواه $w : E \to \mathbb{Z}$ و دو راس $s, t \in V$ را در نظر بگیرید. الگوریتمی با مرتبه زمانی $O(|V|^3)$ ارائه دهید که بتواند کمینه وزن هر مسیری از $s$ به $t$ که شامل حداقل $|V|$ یال است را بیابد.

\مسئله{درخت مینیمال}
در گراف‌های وزن‌دار، «درخت دایگسترای راس $s$» درختی ریشه‌دار از راس $s$ است که فاصله‌ی هرراس در آن (جمع وزن یال‌ها) تا ریشه (راس $s$) برابر کمینه فاصله‌ی آن راس تا ریشه در گراف اصلی است. الگوریتمی از مرتبه زمانی $O((n + m) \log n)$ ارائه دهید که از بین درخت‌های دایگسترای راس $s$، درختی که جمع وزن یال‌های آن کمینه است را پیدا کند.

\مسئله{به وقت ملاقات}
در کشور آبادستان، $n$ شهر وجود دارد که با $m$ جاده دو طرفه به یکدیگر راه دارند. علی در شهر شماره $1$ و اکبر در شهر شماره $n$ است. زمان طی کردن یک جاده برای هر دو نفر دقیقا یک واحد زمان است. علی می‌خواهد به شهر $n$ برود و اکبر می‌خواهد به شهر $1$ برود. آنها نمی‌خواهند در هیچ لحظه‌ای یکدیگر را ببینند اما دیدار در طول یک جاده (زمانی که هر کدام در جهت مخالف یکدیگر حرکت می‌کنند) مجاز است. آنها تا زمانی که هر دو به مقصد نرسند، نمی‌توانند در یک شهر متوقف بمانند و همچنین می‌خواهند در کمترین زمان ممکن هر دو همزمان به مقصد خود برسند. الگوریتمی از $O(n^2 + nm)$ ارائه دهید که کمترین زمان ممکن را بیابد، یا گزارش کند که چنین کاری ممکن نیست.

\مسئله{راس حیاتی}
فرض کنید یک گراف ساده‌ی $n$ راسی
$G=\langle V,E \rangle$ 
به همراه دو راس $s$ و $t$ در $V$ به ما داده شده است.
می‌دانیم که هر مسیر میان این دو راس در گراف $G$ بیش از $\frac{n}{2}$ یال خواهد داشت. الگوریتمی از مرتبه زمانی $O(n + m)$ ارائه دهید که راس $h$ را بیابد که با حذف آن از گراف، دیگر مسیری میان $s$ و $t$ وجود نداشته باشد.


\مسئله{اول بالا بعد پایین}
علی امروز امتحان دی‌اس دارد ولی دیر از خواب بلند شده است، برای همین تصمیم گرفته است برای رفتن به دانشگاه بدود. او مسیری را نیاز دارد که ابتدا کاملا سربالایی و بعد از آن کاملا سرپایینی باشد تا بتواند از انرژی خود به صورت بهینه استفاده کند. حرکت از خانه شروع و به دانشگاه ختم می‌شود. او همچنین یک نقشه که در آن $m$ جاده و $n$ تقاطع وجود دارد را در اختیار دارد (هر جاده میان دو تقاطع واقع شده است).
هر جاده دارای طول مثبت است و هر تقاطع نیز دارای ارتفاع مشخصی است.

\begin{enumerate}
    \item فرض کنید هر جاده یا سربالایی است یا سرپایینی، در این صورت یک الگوریتم بهینه ارائه دهید تا بتواند کوتاه‌ترین مسیر را مطابق توضیحات گفته شده به دست آورد.
    \item یک الگوریتم بهینه ارائه دهید که بتواند این مشکل را که اگر دو تقاطع در یک ارتفاع باشند (به عبارت دیگر جاده مسطح باشد) را حل کند.
\end{enumerate}

\مسئله{قدرت مغول}
یک درخت 
$n$
راسی ریشه‌دار در اختیار داریم.
چنگیز و چونه روی این درخت با یکدیگر بازی می‌کنند.
در ابتدای بازی یک مهره بر روی راس ریشه قرار گرفته است.
بازی به صورت نوبتی انجام می‌شود و هرکس در نوبت خود می‌تواند مهره را از راس کنونی به یکی از راس‌های فرزند آن راس انتقال بدهد.
اگر در نوبت کسی مهره در یک راس برگ باشد (فرزندی نداشته باشد)
او بازنده خواهد شد.
به کمک الگوریتمی در مرتبه زمانی 
$O(n)$
مشخص کنید در صورتی که دو نفر به بهترین شکل ممکن بازی کنند، چه کسی برنده‌ی بازی خواهد شد.

\مسئله{}
گرافی با 
$n$
راس و 
$m$
یال داریم.
\begin{الفبا}
\item حداکثر تعداد یال‌های ممکن را بیابید که با حذف کردن آنها،
از راس
$a$
به راس 
$b$
و از راس 
$c$
به راس 
$d$
همچنان مسیر باقی بماند.
\item به کمک الگوریتمی از مرتبه زمانی $O(n^2 + nm)$ حداکثر تعداد یال‌های ممکن را بیابید که با حذف کردن آنها،
از راس
$a$
به راس 
$b$
مسیری به طول حداکثر 
$l_1$
و از راس 
$c$
به راس 
$d$
مسیری به طول حداکثر 
$l_2$
باقی بماند.
\end{الفبا}

\مسئله{گراف باینری}
گراف وزن‌دار $G(V,E)$ را در نظر بگیرید. وزن هر یال در این گراف برابر صفر یا یک است. حال الگوریتمی ارائه دهید که در زمان $O(m+n)$ فاصله‌ی همه‌ی راس‌ها را تا یک راس مشخص حساب کند ($m$ تعداد یال‌ها و $n$ تعداد راس‌های گراف است).

\پایان{نوشتار}