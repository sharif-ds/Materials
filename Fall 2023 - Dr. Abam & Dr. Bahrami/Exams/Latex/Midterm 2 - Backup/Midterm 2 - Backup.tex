\documentclass[12pt,a4paper]{article}

\usepackage{../commons/course}
\usepackage{listings}

\hidesolutions


\شروع{نوشتار}

\سربرگ{میان‌ترم دوم}{}{زمان آزمون: ۴ بهمن}


‫\مسئله{}
‫
‫یک دایره به شعاع واحد داریم. $n$ نقطه به صورت \textbf{کاملاً تصادفی} در داخل دایره انتخاب می‌کنیم. الگوریتمی با پیچیدگی زمانی $O(n)$ ارائه دهید که بتوان به کمک آن، تمام نقاط را بر حسب فاصله تا مرکز دایره مربط نمود.
‫
‫\مسئله{}
‫
‫درستی عبارات زیر را با ذکر دلیل مشخص کنید.
‫\شروع{شمارش}
‫ \فقره  گره‌های هر درخت دودویی جست و جو را می‌توان با رنگ‌های قرمز و سیاه رنگ کرد طوری که درخت حاصل قرمز-سیاه شود.
‫ \فقره گره‌های هر درخت دودویی جست و جو با $n$ عنصر و ارتفاع حداکثر $2\log n$ را می توان با رنگ‌های قرمز و سیاه رنگ کرد طوری که درخت حاصل قرمز-سیاه شود.
‫ \فقره  یک درخت دودویی جست و جوی کاملا متوازن را می‌توان با رنگ‌های قرمز و سیاه رنگ کرد طوری که درخت حاصل قرمز-سیاه شود.
‫\پایان{شمارش}
‫ 
‫\مسئله{}
‫
‫یک آرایه از n تا عدد داریم. می‌دانیم یک عدد x در این آرایه وجود داره که اکیدا بیشتر از 
‫\lr{n/2}
‫ بار در آرایه، تکرار شده است
‫یک الگوریتم زمان خطی ارائه بدید در این سوال، که این عدد را پیدا کند. 
‫
‫دقت کنید این الگوریتم باید قطعی باشد و استفاده از 
‫\lr{hashmap}
‫برای مثال قابل قبول نیست.
‫
‫ 
‫\مسئله{}
‫
‫یک آرایه از n تا عدد متمایز داریم. می‌دانیم که تعداد نابه‌جایی‌های این آرایه از اردر
‫\lr{O(n)}
‫هستش. در زمان خطی این آرایه را سورت کنید.
‫
‫ 
‫\مسئله{}
‫
‫فرض کنید دو عنصر
‫\lr{a}
‫و
‫\lr{b}
‫از یک درخت دودویی جستجو داده شده است. الگوریتمی پیشنهاد دهید که بزرگ‌ترین عنصر در مسیر دو عنصر داده شده را بیابد. توجه داشته باشید که مسیر بین دو عدد همواره خود اعداد را نیز شامل می‌شود. (پیچیدگی زمانی باید 
‫\lr{O(n)}
‫باشد که
‫\lr{n}
‫ارتفاع درخت است.)
‫
\پایان{نوشتار}