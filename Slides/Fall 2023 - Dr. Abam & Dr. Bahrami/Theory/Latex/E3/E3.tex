‫\documentclass[11pt,a4paper]{article}
‫
‫%\def\hidesols{hide solutions} % uncomment this line to hide solutions
‫\usepackage{../commons/course}
‫
‫
‫\شروع{نوشتار}
‫
‫\سربرگ{تمرین سری سوم}{مرتبه‌ی آماری و مرتب‌سازی}{زمان آزمون: ۱ دی}
‫
‫\مسئله{}
‫
‫داده ساختاری طراحی کنید که درج عنصر، حذف عنصر و پیدا کردن میانه در آن از  $O( \log n) $ باشد. (از داده‌ساختار هیپ کمک بگیرید.)
‫
‫\مسئله{}
‫
‫آرایه‌ای از $ n $ عدد صحیح در بازه‌ی $ 0 $ تا $ n^{2}-1 $ داریم. روشی ارائه دهید که این اعداد را در $ O(n) $ مرتب کند.
‫
‫\مسئله{}
‫
‫تعداد $ n $ چاه نفت در یک نقشه‌ی دوبعدی داریم. چاه نفت $ i $ام در مختصات $ x_{i} $ و  $ y_{i} $ قرار دارد. می‌خواهیم یک لوله‌ی افقی اصلی با مختصات $ y = c $ از بین این چاه‌ها بگذرانیم و هر چاه را با یک لوله‌ی عمودی به این لوله‌ی افقی متصل کنیم. در زمان $ O(n) $ مقدار c را طوری تعیین کنید که مجموع طول لوله‌های عمودی کمینه شود.
‫
‫\مسئله{}
‫
‫تعداد $ n $  دانش‌آموز در یک صف ایستاده‌اند. میزان نارضایتی هر دانش‌آموز برابر با تعداد نفراتی است که جلوتر ایستاده و قدشان از او بلندتر است.
‫
‫\شروع{ابجد}
‫
‫\فقره
‫مجموع نارضایتی دانش‌آموزان در بدترین حالت چقدر خواهد بود؟
‫
‫\فقره
‫با داشتن ترتیب و قد دانش‌آموزان مجموع نارضایتی آن‌ها را در $ O(n\log n) $ به دست‌آورید. 
‫
‫\پایان{ابجد}
‫
‫\مسئله{}
‫
‫دانش‌آموزان یک کلاس را می‌خواهیم با توجه به قدشان به دو گروه تقسیم کنیم به طوری که اعضای گروه اول از همه‌ی اعضای گروه دوم کوتاه‌قدتر باشند. می‌خواهیم گروه اول تا جای ممکن کوچک‌ باشد ولی از طرفی می‌خواهیم مجموع قد افراد گروه اول حداقل برابر نصف مجموع قد همه‌ی افراد باشد. روشی ارائه دهید که در $ O(n) $ این کار را انجام دهد. 
‫
‫\مسئله{}
‫
‫می‌خواهیم از بین $ n $ عدد، $ k $امین کوچکترین عنصر،  $ 2k $امین کوچکترین عنصر و به همین ترتیب تا $ \lfloor \frac{n}{k} \rfloor $امین کوچکترین عنصر را پیدا کنیم. روشی از $ O(n\log \dfrac{n}{k}) $ برای این کار ارائه دهید.
‫
‫\مسئله{}
‫
‫یک آرایه‌ی $n$ عضوی از اعداد صحیح داده شده است. در زمان $O(n)$ عضوی را در صورت وجود پیدا کنید که بیش از $n/3$ بار تکرار شده باشد.
‫
‫\مسئله{}
‫
‫یک ماتریس ۶۴ در ۶۴ داریم که  درایه‌های آن همه ۰ یا ۱  هستند. می‌خواهیم این ماتریس را به صورت 
‫مارپیچی مرتب کنیم یعنی اگر در انتها سطر اول را از چپ به راست به سطر دوم از راست به چپ و ... بچسبانیم یک آرایه‌‌ی ۴۰۹۶ بیتی مرتب از ۰ و ۱ خواهیم داشت. ادعا می‌کنیم که الگوریتم زیر این کار را انجام می‌دهد:
‫
‫\شروع{شمارش}
‫
‫\فقره
‫$k$ بار ایتم‌های ۲ و ۳ را تکرار کن.
‫
‫\فقره
‫همه‌ی سطرها را مستقلا و در جهت خود مرتب کن. یعنی سطرهای فرد را از چپ به راست، و سطرهای زوج را از راست  به چپ مرتب کن.
‫
‫\فقره
‫همه‌ی ستون‌ها را از بالا به پایین مرتب کن.
‫
‫\پایان{شمارش}
‫
‫کم‌ترین مقدار $k$ در بدترین حالت چند است؟ دلیل خود را ذکر کنید.
‫
‫\مسئله{}
‫
‫آرایه‌ی $A$ از $n$ عدد دل‌خواه داده شده است. فرض کنید عملیات
‫$reverse(i, j)$
‫($1 \leq i < j \leq n$)،
‫زیرآرایه‌ی 
‫$A[i..j]$
‫را معکوس می‌کند، یعنی به ازای هر 
‫$0 \leq k \leq j - i$،
‫$A[j - k]$ را درون $A[i + k]$
‫قرار می‌دهد.
‫با چند بار استفاده از این عملیات می‌توان آرایه‌ی $A$ را مرتب کرد؟ دلیل خود را ذکر کنید.
‫
‫\مسئله{}
‫
‫آرایه $A$  شامل $n$ عدد مختلف است. حال می‌خواهیم آرایه $B$ را به این صورت پر کنیم که به ازای هر $i$، $B[i]$ برابر با میانه‌ی اعداد $A[1]$  تا $A[i]$  باشد. الگوریتمی از مرتبه‌ی $O(n\log n)$ برای این مسئله ارائه دهید.
‫
‫\مسئله{}
‫
‫یک دایره به شعاع واحد داریم. $n$ نقطه به صورت \textbf{کاملاً تصادفی} در داخل دایره انتخاب می‌کنیم. الگوریتمی با پیچیدگی زمانی $O(n)$ ارائه دهید که بتوان به کمک آن، تمام نقاط را بر حسب فاصله تا مرکز دایره مربط نمود.
‫
‫\مسئله{}
‫
‫یک جدول $n \times m$ داریم که در هر خانه‌ی آن یک عدد دلخواه وجود دارد. در ابتدا اعداد داخل هر سطر را مرتب می‌کنیم، طوری که هر سطر از چپ به راست به صورت صعودی مرتب شود. سپس اعداد داخل هر ستون را نیز از بالا به پایین به صورت صعودی مرتب می‌کنیم. ثابت کنید بعد از پایان این مرحله، اعداد داخل هر سطر از چپ به راست به صورت صعودی باقی می‌مانند.
‫
‫\مسئله{}
‫
‫یک جدول $n \times n$ داریم که اعداد داخل هر سطر به صورت صعودی از چپ به راست و اعداد داخل هر ستون به صورت صعودی از بالا به پایین نوشته شده‌اند.
‫
‫\شروع{ابجد}
‫
‫\فقره
‫الگوریتمی از مرتبه زمانی $O(n)$ ارائه دهید که اندیس یک خانه در جدول که برابر با $x$ باشد را پیدا کند (در صورت عدم وجود نیز این موضوع اطلاع‌رسانی شود).
‫
‫\فقره
‫الگوریتمی از مرتبه زمانی $O(n)$ ارائه دهید که تعداد خانه‌های جدول که مقدار آن‌ها کم‌تر مساوی $x$ است را بشمارد.
‫
‫\پایان{ابجد}
‫
‫\مسئله{}
‫
‫دستگاه «مجیک‌سورت» یک دستگاه مرتب‌سازی است که حداکثر $n$ توپ در ورودی خود دریافت کرده و می‌تواند در $O(1)$ سنگین‌ترین توپ از بین $n$ توپ داده شده را پیدا کرده و آن را خروجی دهد.
‫
‫\شروع{ابجد}
‫
‫\فقره
‫فرض کنید یک «مجیک‌سورت» و $n^2$ توپ با وزن نامشخص به شما داده می‌شود. الگوریتمی از مرتبه زمانی $O(n^2)$ ارائه دهید که تمام توپ‌ها را بر حسب وزنشان از کم به زیاد مرتب کند.
‫
‫\فقره
‫فرض کنید یک «مجیک‌سورت» و $n^k$ توپ با وزن نامشخص به شما داده می‌شود. الگوریتمی از مرتبه زمانی $O(k \times n^k)$ ارائه دهید که تمام توپ‌ها را بر حسب وزنشان از کم به زیاد مرتب کند.
‫
‫\پایان{ابجد}
‫
‫\مسئله{}
‫
‫آرایه‌ی $A$ از $n$ عدد دل‌خواه متمایز داده شده است و $k$ یک عدد از پیش‌ مشخص است. فرض کنید عملیات $sort(i)$ به ازای $1 \leq i \leq n-k+1$، زیرآرایه‌ی $A[i..i+k-1]$ را مرتب می‌کند. در بدترین حالت چند عملیات $sort$ برای مرتب کردن آرایه‌ی $A$ لازم است؟ 
‫
‫\پایان{نوشتار}